\documentclass[border=5mm]{standalone}
\usepackage{pgfplots}

\begin{document}
\makeatletter
\pgfplotstableread{ % Taken from the python implementation found in the notebook
0 67.917	126.023	282.249	1006.269
1 47.852	85.003	196.066	651.769
2 589.093	990.484	2331.733	8115.294
}\dataset
\begin{tikzpicture}
\begin{axis}[ybar,
        width=13cm,
        height=8cm,
        ymin=0,
        ymax=3500,
        xtick=data,
        xticklabels = {
            1,
            1000,
            1000000
        },
	      xlabel={RBF Complexity},
        ylabel={Fit times (s)},
        major x tick style = {opacity=0},
        minor x tick num = 1,
        minor tick length=2ex,
        nodes near coords,
        point meta=y,
        restrict y to domain*=0:3900,
        after end axis/.code={ % Draw line indicating break
            \draw [ultra thick, white, decoration={snake, amplitude=1pt}, decorate] (rel axis cs:0,1.05) -- (rel axis cs:1,1.05);
        },
        visualization depends on=rawy\as\rawy, % Save the unclipped values
        nodes near coords={%
            \pgfmathprintnumber{\rawy}% Print unclipped values
        },
        every node near coord/.append style={
                anchor=west,
                rotate=90,
        },
        legend style={at={(0.5,-0.20)},
            anchor=north,legend columns=-1},
        axis lines*=left,
        clip=false,
        ]
        \legend{S, M, L, XL},
\addplot[draw=black,fill=blue] table[x index=0,y index=1] \dataset; %Data1
\addplot[draw=black,fill=red] table[x index=0,y index=2] \dataset; %Data2
\addplot[draw=black,fill=green!80!black] table[x index=0,y index=3] \dataset; %Data3
\addplot[draw=black,fill=orange] table[x index=0,y index=4] \dataset; %Data4
\end{axis}
\end{tikzpicture}
\end{document}
